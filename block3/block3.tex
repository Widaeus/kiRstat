% Options for packages loaded elsewhere
\PassOptionsToPackage{unicode}{hyperref}
\PassOptionsToPackage{hyphens}{url}
\PassOptionsToPackage{dvipsnames,svgnames,x11names}{xcolor}
%
\documentclass[
  letterpaper,
  DIV=11,
  numbers=noendperiod]{scrartcl}

\usepackage{amsmath,amssymb}
\usepackage{iftex}
\ifPDFTeX
  \usepackage[T1]{fontenc}
  \usepackage[utf8]{inputenc}
  \usepackage{textcomp} % provide euro and other symbols
\else % if luatex or xetex
  \usepackage{unicode-math}
  \defaultfontfeatures{Scale=MatchLowercase}
  \defaultfontfeatures[\rmfamily]{Ligatures=TeX,Scale=1}
\fi
\usepackage{lmodern}
\ifPDFTeX\else  
    % xetex/luatex font selection
\fi
% Use upquote if available, for straight quotes in verbatim environments
\IfFileExists{upquote.sty}{\usepackage{upquote}}{}
\IfFileExists{microtype.sty}{% use microtype if available
  \usepackage[]{microtype}
  \UseMicrotypeSet[protrusion]{basicmath} % disable protrusion for tt fonts
}{}
\makeatletter
\@ifundefined{KOMAClassName}{% if non-KOMA class
  \IfFileExists{parskip.sty}{%
    \usepackage{parskip}
  }{% else
    \setlength{\parindent}{0pt}
    \setlength{\parskip}{6pt plus 2pt minus 1pt}}
}{% if KOMA class
  \KOMAoptions{parskip=half}}
\makeatother
\usepackage{xcolor}
\setlength{\emergencystretch}{3em} % prevent overfull lines
\setcounter{secnumdepth}{-\maxdimen} % remove section numbering
% Make \paragraph and \subparagraph free-standing
\makeatletter
\ifx\paragraph\undefined\else
  \let\oldparagraph\paragraph
  \renewcommand{\paragraph}{
    \@ifstar
      \xxxParagraphStar
      \xxxParagraphNoStar
  }
  \newcommand{\xxxParagraphStar}[1]{\oldparagraph*{#1}\mbox{}}
  \newcommand{\xxxParagraphNoStar}[1]{\oldparagraph{#1}\mbox{}}
\fi
\ifx\subparagraph\undefined\else
  \let\oldsubparagraph\subparagraph
  \renewcommand{\subparagraph}{
    \@ifstar
      \xxxSubParagraphStar
      \xxxSubParagraphNoStar
  }
  \newcommand{\xxxSubParagraphStar}[1]{\oldsubparagraph*{#1}\mbox{}}
  \newcommand{\xxxSubParagraphNoStar}[1]{\oldsubparagraph{#1}\mbox{}}
\fi
\makeatother

\usepackage{color}
\usepackage{fancyvrb}
\newcommand{\VerbBar}{|}
\newcommand{\VERB}{\Verb[commandchars=\\\{\}]}
\DefineVerbatimEnvironment{Highlighting}{Verbatim}{commandchars=\\\{\}}
% Add ',fontsize=\small' for more characters per line
\usepackage{framed}
\definecolor{shadecolor}{RGB}{241,243,245}
\newenvironment{Shaded}{\begin{snugshade}}{\end{snugshade}}
\newcommand{\AlertTok}[1]{\textcolor[rgb]{0.68,0.00,0.00}{#1}}
\newcommand{\AnnotationTok}[1]{\textcolor[rgb]{0.37,0.37,0.37}{#1}}
\newcommand{\AttributeTok}[1]{\textcolor[rgb]{0.40,0.45,0.13}{#1}}
\newcommand{\BaseNTok}[1]{\textcolor[rgb]{0.68,0.00,0.00}{#1}}
\newcommand{\BuiltInTok}[1]{\textcolor[rgb]{0.00,0.23,0.31}{#1}}
\newcommand{\CharTok}[1]{\textcolor[rgb]{0.13,0.47,0.30}{#1}}
\newcommand{\CommentTok}[1]{\textcolor[rgb]{0.37,0.37,0.37}{#1}}
\newcommand{\CommentVarTok}[1]{\textcolor[rgb]{0.37,0.37,0.37}{\textit{#1}}}
\newcommand{\ConstantTok}[1]{\textcolor[rgb]{0.56,0.35,0.01}{#1}}
\newcommand{\ControlFlowTok}[1]{\textcolor[rgb]{0.00,0.23,0.31}{\textbf{#1}}}
\newcommand{\DataTypeTok}[1]{\textcolor[rgb]{0.68,0.00,0.00}{#1}}
\newcommand{\DecValTok}[1]{\textcolor[rgb]{0.68,0.00,0.00}{#1}}
\newcommand{\DocumentationTok}[1]{\textcolor[rgb]{0.37,0.37,0.37}{\textit{#1}}}
\newcommand{\ErrorTok}[1]{\textcolor[rgb]{0.68,0.00,0.00}{#1}}
\newcommand{\ExtensionTok}[1]{\textcolor[rgb]{0.00,0.23,0.31}{#1}}
\newcommand{\FloatTok}[1]{\textcolor[rgb]{0.68,0.00,0.00}{#1}}
\newcommand{\FunctionTok}[1]{\textcolor[rgb]{0.28,0.35,0.67}{#1}}
\newcommand{\ImportTok}[1]{\textcolor[rgb]{0.00,0.46,0.62}{#1}}
\newcommand{\InformationTok}[1]{\textcolor[rgb]{0.37,0.37,0.37}{#1}}
\newcommand{\KeywordTok}[1]{\textcolor[rgb]{0.00,0.23,0.31}{\textbf{#1}}}
\newcommand{\NormalTok}[1]{\textcolor[rgb]{0.00,0.23,0.31}{#1}}
\newcommand{\OperatorTok}[1]{\textcolor[rgb]{0.37,0.37,0.37}{#1}}
\newcommand{\OtherTok}[1]{\textcolor[rgb]{0.00,0.23,0.31}{#1}}
\newcommand{\PreprocessorTok}[1]{\textcolor[rgb]{0.68,0.00,0.00}{#1}}
\newcommand{\RegionMarkerTok}[1]{\textcolor[rgb]{0.00,0.23,0.31}{#1}}
\newcommand{\SpecialCharTok}[1]{\textcolor[rgb]{0.37,0.37,0.37}{#1}}
\newcommand{\SpecialStringTok}[1]{\textcolor[rgb]{0.13,0.47,0.30}{#1}}
\newcommand{\StringTok}[1]{\textcolor[rgb]{0.13,0.47,0.30}{#1}}
\newcommand{\VariableTok}[1]{\textcolor[rgb]{0.07,0.07,0.07}{#1}}
\newcommand{\VerbatimStringTok}[1]{\textcolor[rgb]{0.13,0.47,0.30}{#1}}
\newcommand{\WarningTok}[1]{\textcolor[rgb]{0.37,0.37,0.37}{\textit{#1}}}

\providecommand{\tightlist}{%
  \setlength{\itemsep}{0pt}\setlength{\parskip}{0pt}}\usepackage{longtable,booktabs,array}
\usepackage{calc} % for calculating minipage widths
% Correct order of tables after \paragraph or \subparagraph
\usepackage{etoolbox}
\makeatletter
\patchcmd\longtable{\par}{\if@noskipsec\mbox{}\fi\par}{}{}
\makeatother
% Allow footnotes in longtable head/foot
\IfFileExists{footnotehyper.sty}{\usepackage{footnotehyper}}{\usepackage{footnote}}
\makesavenoteenv{longtable}
\usepackage{graphicx}
\makeatletter
\def\maxwidth{\ifdim\Gin@nat@width>\linewidth\linewidth\else\Gin@nat@width\fi}
\def\maxheight{\ifdim\Gin@nat@height>\textheight\textheight\else\Gin@nat@height\fi}
\makeatother
% Scale images if necessary, so that they will not overflow the page
% margins by default, and it is still possible to overwrite the defaults
% using explicit options in \includegraphics[width, height, ...]{}
\setkeys{Gin}{width=\maxwidth,height=\maxheight,keepaspectratio}
% Set default figure placement to htbp
\makeatletter
\def\fps@figure{htbp}
\makeatother

\KOMAoption{captions}{tableheading}
\makeatletter
\@ifpackageloaded{caption}{}{\usepackage{caption}}
\AtBeginDocument{%
\ifdefined\contentsname
  \renewcommand*\contentsname{Table of contents}
\else
  \newcommand\contentsname{Table of contents}
\fi
\ifdefined\listfigurename
  \renewcommand*\listfigurename{List of Figures}
\else
  \newcommand\listfigurename{List of Figures}
\fi
\ifdefined\listtablename
  \renewcommand*\listtablename{List of Tables}
\else
  \newcommand\listtablename{List of Tables}
\fi
\ifdefined\figurename
  \renewcommand*\figurename{Figure}
\else
  \newcommand\figurename{Figure}
\fi
\ifdefined\tablename
  \renewcommand*\tablename{Table}
\else
  \newcommand\tablename{Table}
\fi
}
\@ifpackageloaded{float}{}{\usepackage{float}}
\floatstyle{ruled}
\@ifundefined{c@chapter}{\newfloat{codelisting}{h}{lop}}{\newfloat{codelisting}{h}{lop}[chapter]}
\floatname{codelisting}{Listing}
\newcommand*\listoflistings{\listof{codelisting}{List of Listings}}
\makeatother
\makeatletter
\makeatother
\makeatletter
\@ifpackageloaded{caption}{}{\usepackage{caption}}
\@ifpackageloaded{subcaption}{}{\usepackage{subcaption}}
\makeatother

\ifLuaTeX
  \usepackage{selnolig}  % disable illegal ligatures
\fi
\usepackage{bookmark}

\IfFileExists{xurl.sty}{\usepackage{xurl}}{} % add URL line breaks if available
\urlstyle{same} % disable monospaced font for URLs
\hypersetup{
  pdftitle={Inlämningsuppgift Block 3: tabeller, grafer, fördelningar, centrala gränssnittssatsen},
  pdfauthor={Statistiska Metoder med R},
  colorlinks=true,
  linkcolor={blue},
  filecolor={Maroon},
  citecolor={Blue},
  urlcolor={Blue},
  pdfcreator={LaTeX via pandoc}}


\title{Inlämningsuppgift Block 3: tabeller, grafer, fördelningar,
centrala gränssnittssatsen}
\author{Statistiska Metoder med R}
\date{}

\begin{document}
\maketitle


Grupp 10

Namn på dem som deltagit aktivt:

Jacob Widaeus 950729

Rozh Kader

Karl Wärnberg

Beata Rosenberg

\hfill\break

\hfill\break

\textbf{Skriva tabeller}

\textbf{1.1} Ladda ISwR paketet och printa ut de fem första raderna i
data.frame stroke med

\begin{Shaded}
\begin{Highlighting}[]
\FunctionTok{library}\NormalTok{(ISwR)}
\end{Highlighting}
\end{Shaded}

\begin{verbatim}
Warning: package 'ISwR' was built under R version 4.4.1
\end{verbatim}

\begin{Shaded}
\begin{Highlighting}[]
\NormalTok{stroke[}\DecValTok{1}\SpecialCharTok{:}\DecValTok{5}\NormalTok{,]}
\end{Highlighting}
\end{Shaded}

\begin{verbatim}
     sex       died       dstr age dgn coma diab minf han  dead   obsmonths
1   Male 1991-01-07 1991-01-02  76 INF   No   No  Yes  No  TRUE  0.16339869
2   Male       <NA> 1991-01-03  58 INF   No   No   No  No FALSE 59.60784314
3   Male 1991-06-02 1991-01-08  74 INF   No   No  Yes Yes  TRUE  4.73856209
4 Female 1991-01-13 1991-01-11  77 ICH   No  Yes   No Yes  TRUE  0.06535948
5 Female       <NA> 1991-01-13  76 INF   No  Yes   No Yes FALSE 59.28104575
\end{verbatim}

Varför står det ett ensamt komma efter 1:5?

För att markera att det gäller samtliga kolumner.

\textbf{1.2} Använd funktionen names() för att ta ut en lista på
variabler i stroke

\begin{Shaded}
\begin{Highlighting}[]
\FunctionTok{names}\NormalTok{(stroke)}
\end{Highlighting}
\end{Shaded}

\begin{verbatim}
 [1] "sex"       "died"      "dstr"      "age"       "dgn"       "coma"     
 [7] "diab"      "minf"      "han"       "dead"      "obsmonths"
\end{verbatim}

\textbf{1.3} Är age tillgänglig? Testa:

\begin{Shaded}
\begin{Highlighting}[]
\NormalTok{age}
\end{Highlighting}
\end{Shaded}

\begin{verbatim}
Error in eval(expr, envir, enclos): object 'age' not found
\end{verbatim}

\textbf{1.4} Använd attach för att namnen i stroke ska blir tillgängliga
för R

\begin{Shaded}
\begin{Highlighting}[]
\FunctionTok{attach}\NormalTok{(stroke)}
\end{Highlighting}
\end{Shaded}

Testa igen om age är tillgänglig:

\begin{Shaded}
\begin{Highlighting}[]
\FunctionTok{print}\NormalTok{(age[}\DecValTok{1}\SpecialCharTok{:}\DecValTok{20}\NormalTok{])}
\end{Highlighting}
\end{Shaded}

\begin{verbatim}
 [1] 76 58 74 77 76 48 81 53 73 69 86 79 69 58 71 84 63 85 81 77
\end{verbatim}

\textbf{1.5} Vad är tanken med attach egentligen? Vad finns det för
fördel med att R inte automatiskt ser namnen inne i ett objekt? Och om
det nu är så bra, finns det något sätt att få R att glömma namnen i ett
objekt? (ISwR sid.36)

Tanken med attach är att göra variabler ``direkta'' i en workspace utan
att behöva referera till data frame. I större projekt där man använder
många data set och frames och kanske listor underlättar namngivning etc
utan attach. detach() kan användas för att återställa
variabeltillgängligheten.

\textbf{1.6} Använd tapply() för att beräkna medelvärde för variabeln
age, uppdelat på patienter som varit i koma eller inte efter sin stroke.

\begin{Shaded}
\begin{Highlighting}[]
\FunctionTok{tapply}\NormalTok{(stroke}\SpecialCharTok{$}\NormalTok{age, stroke}\SpecialCharTok{$}\NormalTok{coma, mean, }\AttributeTok{na.rm =} \ConstantTok{TRUE}\NormalTok{)}
\end{Highlighting}
\end{Shaded}

\begin{verbatim}
      No      Yes 
69.59463 72.18750 
\end{verbatim}

\textbf{1.7} Använd tapply() för att beräkna medianålder för patienten
uppdelat på olika diagnos, dgn.

\begin{Shaded}
\begin{Highlighting}[]
\FunctionTok{tapply}\NormalTok{(stroke}\SpecialCharTok{$}\NormalTok{age, stroke}\SpecialCharTok{$}\NormalTok{dgn, median, }\AttributeTok{na.rm =} \ConstantTok{TRUE}\NormalTok{)}
\end{Highlighting}
\end{Shaded}

\begin{verbatim}
ICH  ID INF SAH 
 68  81  70  60 
\end{verbatim}

\textbf{1.8} Använd funktionen table() för att göra en korstabell över
variablerna dgn och sex i stroke

\begin{Shaded}
\begin{Highlighting}[]
\FunctionTok{table}\NormalTok{(stroke}\SpecialCharTok{$}\NormalTok{sex, stroke}\SpecialCharTok{$}\NormalTok{dgn)}
\end{Highlighting}
\end{Shaded}

\begin{verbatim}
        
         ICH  ID INF SAH
  Female  48 140 295  27
  Male    31  62 206  20
\end{verbatim}

\textbf{Använd valfria data i stroke för att skapa grafer:}

\textbf{2.1} stapeldiagram

\begin{Shaded}
\begin{Highlighting}[]
\FunctionTok{barplot}\NormalTok{(}\FunctionTok{table}\NormalTok{(stroke}\SpecialCharTok{$}\NormalTok{sex))}
\end{Highlighting}
\end{Shaded}

\includegraphics{block3_files/figure-pdf/unnamed-chunk-9-1.pdf}

Stapeldiagram lämpar sig för att skildra antal i kategoriska variabler.

\textbf{2.2} histogram

\begin{Shaded}
\begin{Highlighting}[]
\FunctionTok{hist}\NormalTok{(stroke}\SpecialCharTok{$}\NormalTok{age)}
\end{Highlighting}
\end{Shaded}

\includegraphics{block3_files/figure-pdf/unnamed-chunk-10-1.pdf}

Histogram lämpar sig för att skildra antal i en kontinuerlig variabel
och utvärdera fördelning i en population.

\textbf{2.3} box plot

\begin{Shaded}
\begin{Highlighting}[]
\FunctionTok{boxplot}\NormalTok{(stroke}\SpecialCharTok{$}\NormalTok{obsmonths }\SpecialCharTok{\textasciitilde{}}\NormalTok{ stroke}\SpecialCharTok{$}\NormalTok{dead)}
\end{Highlighting}
\end{Shaded}

\includegraphics{block3_files/figure-pdf/unnamed-chunk-11-1.pdf}

Box plot lämpar sig bäst för att skildra intravariabel spridning i
jämförelse med andra variabler.

\textbf{2.4} spridningsdiagram (scatter plot)

\begin{Shaded}
\begin{Highlighting}[]
\FunctionTok{plot}\NormalTok{(stroke}\SpecialCharTok{$}\NormalTok{died, stroke}\SpecialCharTok{$}\NormalTok{age)}
\end{Highlighting}
\end{Shaded}

\includegraphics{block3_files/figure-pdf/unnamed-chunk-12-1.pdf}

Lämpar sig bäst för att utröna kluster, mönster och samband mellan två
kontinuerliga variabler.

\textbf{2.5} cirkeldiagram

\begin{Shaded}
\begin{Highlighting}[]
\FunctionTok{pie}\NormalTok{(}\FunctionTok{table}\NormalTok{(stroke}\SpecialCharTok{$}\NormalTok{dgn))}
\end{Highlighting}
\end{Shaded}

\includegraphics{block3_files/figure-pdf/unnamed-chunk-13-1.pdf}

Lämpar sig bäst för att utvärdera antalet inom olika kategoriska
variabler, i jämförelse till varandra och helheten.

Förutom att redovisa din kod och klippa in den resulterande grafen vill
jag att du kommenterar kort vad varje typ av graf är lämplig för att
illustrera.

Kanske valde du att besvara fråga 2.1 med den enkla koden
plot(table(dgn))

I så fall får du godkänt. Men om du svarat lika enkelt på alla frågor
2.1-5 så vill jag att du utforskar några möjligheter att göra en mer
avancerad graf.

\textbf{3.1} Rita plot(table(dgn)) men lägg till något flärdfullt som
bakgrundsfärg, titelrad, tjockare staplar.

\begin{Shaded}
\begin{Highlighting}[]
\CommentTok{\# Create a table of counts for the categorical variable \textquotesingle{}dgn\textquotesingle{}}
\NormalTok{dgn\_counts }\OtherTok{\textless{}{-}} \FunctionTok{table}\NormalTok{(stroke}\SpecialCharTok{$}\NormalTok{dgn)}

\CommentTok{\# Create a bar plot with enhanced formatting}
\FunctionTok{barplot}\NormalTok{(}
\NormalTok{  dgn\_counts,}
  \AttributeTok{main =} \StringTok{"Fördelning av diagnoser"}\NormalTok{,}
  \AttributeTok{xlab =} \StringTok{"Diagnoser"}\NormalTok{,}
  \AttributeTok{ylab =} \StringTok{"Frekvens"}\NormalTok{,}
  \AttributeTok{col =} \StringTok{"skyblue"}\NormalTok{,}
  \AttributeTok{border =} \StringTok{"white"}
\NormalTok{)}
\end{Highlighting}
\end{Shaded}

\includegraphics{block3_files/figure-pdf/unnamed-chunk-14-1.pdf}

\textbf{Välj, beräkna och motivera val av centralmått och spridningsmått
för dessa variabler:}

\textbf{4.1} age i stroke

\begin{Shaded}
\begin{Highlighting}[]
\NormalTok{age\_median }\OtherTok{\textless{}{-}} \FunctionTok{median}\NormalTok{(stroke}\SpecialCharTok{$}\NormalTok{age)}
\NormalTok{age\_iqr }\OtherTok{\textless{}{-}} \FunctionTok{IQR}\NormalTok{(stroke}\SpecialCharTok{$}\NormalTok{age)}
\NormalTok{age\_lower }\OtherTok{\textless{}{-}} \FunctionTok{quantile}\NormalTok{(stroke}\SpecialCharTok{$}\NormalTok{age, }\FloatTok{0.25}\NormalTok{)}
\NormalTok{age\_upper }\OtherTok{\textless{}{-}} \FunctionTok{quantile}\NormalTok{(stroke}\SpecialCharTok{$}\NormalTok{age, }\FloatTok{0.75}\NormalTok{)}

\FunctionTok{hist}\NormalTok{(stroke}\SpecialCharTok{$}\NormalTok{age)}

\FunctionTok{abline}\NormalTok{(}\AttributeTok{v =}\NormalTok{ age\_median)}
\FunctionTok{abline}\NormalTok{(}\AttributeTok{v =}\NormalTok{ age\_lower)}
\FunctionTok{abline}\NormalTok{(}\AttributeTok{v =}\NormalTok{ age\_upper)}
\end{Highlighting}
\end{Shaded}

\includegraphics{block3_files/figure-pdf/unnamed-chunk-15-1.pdf}

Här valde jag att beräkna median och IQR på grund av vänsterskewed data.

\textbf{4.2} dgn i stroke

\begin{Shaded}
\begin{Highlighting}[]
\FunctionTok{plot}\NormalTok{(stroke}\SpecialCharTok{$}\NormalTok{dgn)}
\end{Highlighting}
\end{Shaded}

\includegraphics{block3_files/figure-pdf/unnamed-chunk-16-1.pdf}

\begin{Shaded}
\begin{Highlighting}[]
\FunctionTok{print}\NormalTok{(}\FunctionTok{paste}\NormalTok{(}\StringTok{"Mode av stroke$dgn är"}\NormalTok{, }\FunctionTok{names}\NormalTok{(}\FunctionTok{sort}\NormalTok{(}\FunctionTok{table}\NormalTok{(stroke}\SpecialCharTok{$}\NormalTok{dgn), }\AttributeTok{decreasing =} \ConstantTok{TRUE}\NormalTok{)[}\DecValTok{1}\NormalTok{])))}
\end{Highlighting}
\end{Shaded}

\begin{verbatim}
[1] "Mode av stroke$dgn är INF"
\end{verbatim}

Bästa centralmåttet av kategoriska variabler är mode.

\textbf{4.3} rnorm(10000)

Om du vet vad rnorm betyder -- testa annars ?rnorm() -- så är det lätt
att argumentera för det rätta svaret. Men jag vill påminna här att du
även kan titta på data innan du bestämmer dig. Kanske dessa metoder kan
vara till nytta: hist(rnorm(10000)) boxplot(rnorm(10000))

\begin{Shaded}
\begin{Highlighting}[]
\NormalTok{data }\OtherTok{\textless{}{-}} \FunctionTok{rnorm}\NormalTok{(}\DecValTok{10000}\NormalTok{)}
\FunctionTok{hist}\NormalTok{(data)}

\FunctionTok{abline}\NormalTok{(}\AttributeTok{v =} \FunctionTok{mean}\NormalTok{(data))}
\FunctionTok{abline}\NormalTok{(}\AttributeTok{v =} \FunctionTok{mean}\NormalTok{(data) }\SpecialCharTok{{-}} \FunctionTok{sd}\NormalTok{(data))}
\FunctionTok{abline}\NormalTok{(}\AttributeTok{v =} \FunctionTok{mean}\NormalTok{(data) }\SpecialCharTok{+} \FunctionTok{sd}\NormalTok{(data))}
\end{Highlighting}
\end{Shaded}

\includegraphics{block3_files/figure-pdf/unnamed-chunk-17-1.pdf}

I en normalfördelad data hamnar medelvärdet centralt och spridningen kan
karaktäriseras standardavvikelser. I detta fall skildrar jag detta med
en vertikal linje 1 SD från medelvärdet.

\textbf{Fördelningar}

För att ha glädje av avsnittet fördelningar i inlämningsuppgiften måste
du ha grunderna klara för dig. Läs i Lind eller någon anna text om
diskreta och kontinuerliga fördelningar. Du kan fokusera på exemplen
binomialfördelning (sid. 184-193) och normalfördelning (214-229).

\textbf{Bionomialfördelningen}

Passar för att beskriva fördelningar där man upprepar ett experimentet
ett antal gånger (i R heter det size) Experimentet kan få två olika
utfall, dessa brukar betecknas success och failure. Sannolikheten för
success (i R heter det prob) förblir densamma vid varje nytt försök.

Exempel: Två patienter får en behandling som har sannolikheten P(bota) =
0,7 att bota sjukdomen.

\textbf{5.1} Använd min kod nedan för att med binomialfördelningen
beräkna och plotta sannolikheterna att 0 eller 1 eller 2 patienter
botas. Lägg till en lämplig titel till grafen.

\begin{Shaded}
\begin{Highlighting}[]
\NormalTok{x }\OtherTok{\textless{}{-}} \DecValTok{0}\SpecialCharTok{:}\DecValTok{2}
\FunctionTok{plot}\NormalTok{(x, }\FunctionTok{dbinom}\NormalTok{(x, }\DecValTok{2}\NormalTok{, }\FloatTok{0.7}\NormalTok{), }\AttributeTok{type =} \StringTok{"h"}\NormalTok{, }\AttributeTok{col =} \StringTok{"red"}\NormalTok{, }\AttributeTok{lwd=}\DecValTok{10}\NormalTok{, }\AttributeTok{main =} \StringTok{"Binomial fördelning"}\NormalTok{)}
\end{Highlighting}
\end{Shaded}

\includegraphics{block3_files/figure-pdf/unnamed-chunk-18-1.pdf}

Det är bra att vid ett tillfälle beräkna sannolikheterna för varje
utfall för hand. Formeln finner du på sid 57 i ISwR. Här följer formeln
med värdena för utfallet 0 tillfrisknar ifyllda.

\begin{Shaded}
\begin{Highlighting}[]
\NormalTok{n }\OtherTok{\textless{}{-}} \DecValTok{2}
\NormalTok{p }\OtherTok{\textless{}{-}} \FloatTok{0.7}

\CommentTok{\# Beräkna sannolikheterna}
\NormalTok{prob\_0 }\OtherTok{\textless{}{-}} \FunctionTok{dbinom}\NormalTok{(}\DecValTok{0}\NormalTok{, n, p)}
\NormalTok{prob\_1 }\OtherTok{\textless{}{-}} \FunctionTok{dbinom}\NormalTok{(}\DecValTok{1}\NormalTok{, n, p)}
\NormalTok{prob\_2 }\OtherTok{\textless{}{-}} \FunctionTok{dbinom}\NormalTok{(}\DecValTok{2}\NormalTok{, n, p)}

\CommentTok{\# print}
\FunctionTok{print}\NormalTok{(}\FunctionTok{paste}\NormalTok{(}\StringTok{"P(x=0) ="}\NormalTok{, prob\_0))}
\end{Highlighting}
\end{Shaded}

\begin{verbatim}
[1] "P(x=0) = 0.09"
\end{verbatim}

\begin{Shaded}
\begin{Highlighting}[]
\FunctionTok{print}\NormalTok{(}\FunctionTok{paste}\NormalTok{(}\StringTok{"P(x=1) ="}\NormalTok{, prob\_1))}
\end{Highlighting}
\end{Shaded}

\begin{verbatim}
[1] "P(x=1) = 0.42"
\end{verbatim}

\begin{Shaded}
\begin{Highlighting}[]
\FunctionTok{print}\NormalTok{(}\FunctionTok{paste}\NormalTok{(}\StringTok{"P(x=2) ="}\NormalTok{, prob\_2))}
\end{Highlighting}
\end{Shaded}

\begin{verbatim}
[1] "P(x=2) = 0.49"
\end{verbatim}

\textbf{5.2} Beräkna i R med formeln ovan sannolikheterna att 0 eller 1
eller 2 patienter botas, rita en graf. Använd funktionen choose(). Blev
det samma resultat som i 5.1?

\begin{Shaded}
\begin{Highlighting}[]
\CommentTok{\# Beräkna sannolikheterna med ovan formel}
\NormalTok{prob\_0 }\OtherTok{\textless{}{-}} \FunctionTok{choose}\NormalTok{(n, }\DecValTok{0}\NormalTok{) }\SpecialCharTok{*}\NormalTok{ p}\SpecialCharTok{\^{}}\DecValTok{0} \SpecialCharTok{*}\NormalTok{ (}\DecValTok{1} \SpecialCharTok{{-}}\NormalTok{ p)}\SpecialCharTok{\^{}}\NormalTok{(n }\SpecialCharTok{{-}} \DecValTok{0}\NormalTok{)}
\NormalTok{prob\_1 }\OtherTok{\textless{}{-}} \FunctionTok{choose}\NormalTok{(n, }\DecValTok{1}\NormalTok{) }\SpecialCharTok{*}\NormalTok{ p}\SpecialCharTok{\^{}}\DecValTok{1} \SpecialCharTok{*}\NormalTok{ (}\DecValTok{1} \SpecialCharTok{{-}}\NormalTok{ p)}\SpecialCharTok{\^{}}\NormalTok{(n }\SpecialCharTok{{-}} \DecValTok{1}\NormalTok{)}
\NormalTok{prob\_2 }\OtherTok{\textless{}{-}} \FunctionTok{choose}\NormalTok{(n, }\DecValTok{2}\NormalTok{) }\SpecialCharTok{*}\NormalTok{ p}\SpecialCharTok{\^{}}\DecValTok{2} \SpecialCharTok{*}\NormalTok{ (}\DecValTok{1} \SpecialCharTok{{-}}\NormalTok{ p)}\SpecialCharTok{\^{}}\NormalTok{(n }\SpecialCharTok{{-}} \DecValTok{2}\NormalTok{)}

\CommentTok{\# Skapa en vektor}
\NormalTok{probs }\OtherTok{\textless{}{-}} \FunctionTok{c}\NormalTok{(prob\_0, prob\_1, prob\_2)}

\CommentTok{\# Plot}
\FunctionTok{barplot}\NormalTok{(probs, }\AttributeTok{names.arg =} \DecValTok{0}\SpecialCharTok{:}\DecValTok{2}\NormalTok{, }\AttributeTok{main =} \StringTok{"Sannolikhet"}\NormalTok{, }\AttributeTok{xlab =} \StringTok{"Antal tillfrisknade"}\NormalTok{, }\AttributeTok{ylab =} \StringTok{"Sannolikhet"}\NormalTok{, }\AttributeTok{col =} \StringTok{"lightblue"}\NormalTok{, }\AttributeTok{border =} \StringTok{"white"}\NormalTok{, }\AttributeTok{ylim =} \FunctionTok{c}\NormalTok{(}\DecValTok{0}\NormalTok{,}\FloatTok{0.75}\NormalTok{))}
\end{Highlighting}
\end{Shaded}

\includegraphics{block3_files/figure-pdf/unnamed-chunk-20-1.pdf}

För att lösa uppgiften kan du behöva komplettera texten i ISwR med en
annan text om binomialfördelningen. Om du har läst tidigare räcker det
kanske att läsa igenom formler som jag klippt ur wikipedia binomial
distribution

\textbf{5.3} Sex patienter får samma behandling. Sannolikheten att
tillfriskna har visat sig vara 80\%.

Rita ett stapeldiagram som visar sannolikheten för varje tänkbart utfall
från noll patienter tillfrisknar till sex patienter tillfrisknar. Använd
dbinom().

\begin{Shaded}
\begin{Highlighting}[]
\CommentTok{\# Skriver funktion då det kan itereras}
\NormalTok{prob }\OtherTok{\textless{}{-}} \ControlFlowTok{function}\NormalTok{(n, p) \{}
  \CommentTok{\# Initiate df}
\NormalTok{  prob }\OtherTok{\textless{}{-}} \FunctionTok{data.frame}\NormalTok{(}\AttributeTok{amount =} \FunctionTok{integer}\NormalTok{(), }\AttributeTok{probability =} \FunctionTok{numeric}\NormalTok{())}
  \CommentTok{\# for loop, for every iteration calculate and append}
  \ControlFlowTok{for}\NormalTok{ (i }\ControlFlowTok{in} \DecValTok{1}\SpecialCharTok{:}\NormalTok{n) \{}
\NormalTok{    prob }\OtherTok{\textless{}{-}} \FunctionTok{rbind}\NormalTok{(prob, }\FunctionTok{data.frame}\NormalTok{(}\AttributeTok{amount =}\NormalTok{ i, }\AttributeTok{probability =} \FunctionTok{dbinom}\NormalTok{(i, n, p)))}
\NormalTok{  \}}
  \FunctionTok{return}\NormalTok{(prob)}
\NormalTok{\}}

\CommentTok{\# Beräkna sannolikheter}
\NormalTok{probabilities }\OtherTok{\textless{}{-}} \FunctionTok{prob}\NormalTok{(}\DecValTok{6}\NormalTok{, }\FloatTok{0.8}\NormalTok{)}

\CommentTok{\# Plot}
\FunctionTok{barplot}\NormalTok{(probabilities}\SpecialCharTok{$}\NormalTok{probability, }\AttributeTok{names.arg =}\NormalTok{ probabilities}\SpecialCharTok{$}\NormalTok{amount, }\AttributeTok{ylim =} \FunctionTok{c}\NormalTok{(}\DecValTok{0}\NormalTok{, }\FloatTok{0.5}\NormalTok{))}
\end{Highlighting}
\end{Shaded}

\includegraphics{block3_files/figure-pdf/unnamed-chunk-21-1.pdf}

\textbf{5.4} Beräkna sannolikheten att exakt två patienter tillfrisknar.

\begin{Shaded}
\begin{Highlighting}[]
\CommentTok{\# Det finns redan en funktion skriven}

\NormalTok{frisk\_2 }\OtherTok{\textless{}{-}} \FunctionTok{prob}\NormalTok{(}\DecValTok{2}\NormalTok{, }\FloatTok{0.8}\NormalTok{)}

\FunctionTok{print}\NormalTok{(}\FunctionTok{paste}\NormalTok{(}\StringTok{"Sannolikheten att 2 patienter tillfrisknar är"}\NormalTok{, frisk\_2[}\DecValTok{2}\NormalTok{,}\DecValTok{2}\NormalTok{]}\SpecialCharTok{*}\DecValTok{100}\NormalTok{,}\StringTok{"\%"}\NormalTok{))}
\end{Highlighting}
\end{Shaded}

\begin{verbatim}
[1] "Sannolikheten att 2 patienter tillfrisknar är 64 %"
\end{verbatim}

\textbf{5.5} Beräkna sannolikheten att minst 5 patienter tillfrisknar.

\begin{Shaded}
\begin{Highlighting}[]
\NormalTok{frisk\_2 }\OtherTok{\textless{}{-}} \FunctionTok{prob}\NormalTok{(}\DecValTok{5}\NormalTok{, }\FloatTok{0.8}\NormalTok{)}

\FunctionTok{print}\NormalTok{(}\FunctionTok{paste}\NormalTok{(}\StringTok{"Sannolikheten att 5 patienter tillfrisknar är"}\NormalTok{, }\FunctionTok{round}\NormalTok{(frisk\_2[}\DecValTok{5}\NormalTok{,}\DecValTok{2}\NormalTok{]}\SpecialCharTok{*}\DecValTok{100}\NormalTok{),}\StringTok{"\%"}\NormalTok{))}
\end{Highlighting}
\end{Shaded}

\begin{verbatim}
[1] "Sannolikheten att 5 patienter tillfrisknar är 33 %"
\end{verbatim}

\textbf{Normalfördelnigen}

Normalfördelnigen är ett mycket användbart verktyg inom statistiken.

\textbf{6.1} Rita en standardiserad normalfördelning med medelvärdet
noll och standardavvikelsen ett med mitt kodförslag nedan:

x \textless- seq(-4,4,0.05)

plot (dnorm(x))

eller

curve(dnorm(x), from= -4, to=4)

\emph{Funktionen dnorm(x), som i density function, returnerar
sannolikheten att i ett experiment få ett ufallet x eller mycket nära x}

\begin{Shaded}
\begin{Highlighting}[]
\NormalTok{x }\OtherTok{\textless{}{-}} \FunctionTok{seq}\NormalTok{(}\SpecialCharTok{{-}}\DecValTok{4}\NormalTok{,}\DecValTok{4}\NormalTok{,}\FloatTok{0.05}\NormalTok{)}

\FunctionTok{plot}\NormalTok{ (}\FunctionTok{dnorm}\NormalTok{(x))}
\end{Highlighting}
\end{Shaded}

\includegraphics{block3_files/figure-pdf/unnamed-chunk-24-1.pdf}

\textbf{6.2} Rita en graf som beskriver fördelningen av blodtryck i mmHg
för människor. Variabeln ska vara normalfödelad med medelvärdet 90 och
standardavvikelsen 10.

\begin{Shaded}
\begin{Highlighting}[]
\FunctionTok{hist}\NormalTok{(}\FunctionTok{rnorm}\NormalTok{(}\DecValTok{100}\NormalTok{, }\AttributeTok{mean=}\DecValTok{90}\NormalTok{, }\AttributeTok{sd=}\DecValTok{10}\NormalTok{))}
\end{Highlighting}
\end{Shaded}

\includegraphics{block3_files/figure-pdf/unnamed-chunk-25-1.pdf}

\emph{Funktionen pnorm(x), som i probability function, returnerar
sannolikheten att i ett experiment få ett utfall x eller lägre än x.}

Använd pnorm() för att beräkna andelen människor i världsbefolkningen
som har ett blodtryck:

\begin{Shaded}
\begin{Highlighting}[]
\CommentTok{\# Skriver en funktion då repeterande uppgift}
\NormalTok{blodtryck }\OtherTok{\textless{}{-}} \ControlFlowTok{function}\NormalTok{(limit, mean, sd, }\AttributeTok{hi =} \ConstantTok{TRUE}\NormalTok{) \{}
\NormalTok{  prob }\OtherTok{\textless{}{-}} \FunctionTok{pnorm}\NormalTok{(limit, }\AttributeTok{mean =}\NormalTok{ mean, }\AttributeTok{sd =}\NormalTok{ sd)}
  \ControlFlowTok{if}\NormalTok{ (hi)}
\NormalTok{    prob }\OtherTok{\textless{}{-}}\NormalTok{ (}\DecValTok{1}\SpecialCharTok{{-}}\NormalTok{prob)}\SpecialCharTok{*}\DecValTok{100}
  \ControlFlowTok{else}
\NormalTok{    prob }\OtherTok{\textless{}{-}}\NormalTok{ prob}\SpecialCharTok{*}\DecValTok{100}

  \FunctionTok{return}\NormalTok{(prob)}
\NormalTok{\}}
\end{Highlighting}
\end{Shaded}

\textbf{6.3} 80 eller lägre

\begin{Shaded}
\begin{Highlighting}[]
\FunctionTok{print}\NormalTok{(}\FunctionTok{blodtryck}\NormalTok{(}\DecValTok{80}\NormalTok{, }\AttributeTok{mean=}\DecValTok{90}\NormalTok{, }\AttributeTok{sd=}\DecValTok{10}\NormalTok{, }\AttributeTok{hi =} \ConstantTok{FALSE}\NormalTok{))}
\end{Highlighting}
\end{Shaded}

\begin{verbatim}
[1] 15.86553
\end{verbatim}

\textbf{6.4} 100 eller lägre

\begin{Shaded}
\begin{Highlighting}[]
\FunctionTok{print}\NormalTok{(}\FunctionTok{blodtryck}\NormalTok{(}\DecValTok{100}\NormalTok{, }\AttributeTok{mean=}\DecValTok{90}\NormalTok{, }\AttributeTok{sd=}\DecValTok{10}\NormalTok{, }\AttributeTok{hi =} \ConstantTok{FALSE}\NormalTok{))}
\end{Highlighting}
\end{Shaded}

\begin{verbatim}
[1] 84.13447
\end{verbatim}

\textbf{6.5} högre än 100

\begin{Shaded}
\begin{Highlighting}[]
\FunctionTok{print}\NormalTok{(}\FunctionTok{blodtryck}\NormalTok{(}\DecValTok{100}\NormalTok{, }\AttributeTok{mean=}\DecValTok{90}\NormalTok{, }\AttributeTok{sd=}\DecValTok{10}\NormalTok{, }\AttributeTok{hi =} \ConstantTok{TRUE}\NormalTok{))}
\end{Highlighting}
\end{Shaded}

\begin{verbatim}
[1] 15.86553
\end{verbatim}

\textbf{6.6} högre än 90 (stanna till här och tänk efter om resultatet
för 6.6 verkar stämma, kommentera)

\begin{Shaded}
\begin{Highlighting}[]
\FunctionTok{print}\NormalTok{(}\FunctionTok{blodtryck}\NormalTok{(}\DecValTok{90}\NormalTok{, }\AttributeTok{mean=}\DecValTok{90}\NormalTok{, }\AttributeTok{sd=}\DecValTok{10}\NormalTok{, }\AttributeTok{hi =} \ConstantTok{TRUE}\NormalTok{))}
\end{Highlighting}
\end{Shaded}

\begin{verbatim}
[1] 50
\end{verbatim}

Eftersom medelvärdet av normalfördelning ska vara 90 bör då 50\% vara
över 90.

\textbf{6.7} Av 50000 svenska män födda 1980, hur många är 190 cm eller
längre? Utgå ifrån medellängd 180 cm och standardavvikelse 7 cm.
(Värdena är påhittade.)

\begin{Shaded}
\begin{Highlighting}[]
\FunctionTok{print}\NormalTok{((}\DecValTok{1}\SpecialCharTok{{-}}\FunctionTok{pnorm}\NormalTok{(}\DecValTok{190}\NormalTok{, }\AttributeTok{mean =} \DecValTok{180}\NormalTok{, }\AttributeTok{sd =} \DecValTok{7}\NormalTok{)))}
\end{Highlighting}
\end{Shaded}

\begin{verbatim}
[1] 0.07656373
\end{verbatim}

\textbf{6.8} Hur många män födda 1980 är mellan 180 och 190 cm långa?

\begin{Shaded}
\begin{Highlighting}[]
\FunctionTok{print}\NormalTok{(}\FunctionTok{pnorm}\NormalTok{(}\DecValTok{190}\NormalTok{, }\AttributeTok{mean =} \DecValTok{180}\NormalTok{, }\AttributeTok{sd =} \DecValTok{7}\NormalTok{) }\SpecialCharTok{{-}} \FunctionTok{pnorm}\NormalTok{(}\DecValTok{180}\NormalTok{, }\AttributeTok{mean =} \DecValTok{180}\NormalTok{, }\AttributeTok{sd =} \DecValTok{7}\NormalTok{))}
\end{Highlighting}
\end{Shaded}

\begin{verbatim}
[1] 0.4234363
\end{verbatim}

\emph{Funktionen qnorm(), som i quantile function, kan du använda om du
söker ett värde för en normalfördelad variabel som är högre eller lika
med en viss andel av alla observationer.}

Nedan har jag använt qnorm() för att beräkna det intervall av blodtryck
som innesluter 95\% av befolkningen.

\textgreater{} qnorm(c(0.025, 0.975), 90, 10)

{[}1{]} 70.40036 109.59964

\textbf{6.9} Justera min kod så att R returnerar blodtrycksintervallet
angett i heltal.

\begin{Shaded}
\begin{Highlighting}[]
\FunctionTok{round}\NormalTok{(}\FunctionTok{qnorm}\NormalTok{(}\FunctionTok{c}\NormalTok{(}\FloatTok{0.025}\NormalTok{, }\FloatTok{0.975}\NormalTok{), }\DecValTok{90}\NormalTok{, }\DecValTok{10}\NormalTok{))}
\end{Highlighting}
\end{Shaded}

\begin{verbatim}
[1]  70 110
\end{verbatim}

Nedan har jag ritat en graf som visar fördelningen av blodtryck i
befolkningen. Medel 90 mmHg, standardavvikelse 10. Dessutom har jag
ritat in gränserna som innesluter 95\% av befolkningen.

\textbf{6.10} Återskapa min graf med R kod.

\begin{Shaded}
\begin{Highlighting}[]
\CommentTok{\# Definierar punkter}
\NormalTok{x }\OtherTok{\textless{}{-}} \FunctionTok{seq}\NormalTok{(}\DecValTok{60}\NormalTok{, }\DecValTok{120}\NormalTok{, }\FloatTok{0.1}\NormalTok{)}

\CommentTok{\# Beräknar värden}
\NormalTok{blodtryck }\OtherTok{\textless{}{-}} \FunctionTok{dnorm}\NormalTok{(x, }\AttributeTok{mean =} \DecValTok{90}\NormalTok{, }\AttributeTok{sd =} \DecValTok{10}\NormalTok{)}

\NormalTok{ci }\OtherTok{\textless{}{-}} \FunctionTok{round}\NormalTok{(}\FunctionTok{qnorm}\NormalTok{(}\FunctionTok{c}\NormalTok{(}\FloatTok{0.025}\NormalTok{, }\FloatTok{0.975}\NormalTok{), }\DecValTok{90}\NormalTok{, }\DecValTok{10}\NormalTok{))}

\CommentTok{\# Plot}
\FunctionTok{plot}\NormalTok{(x, blodtryck, }\AttributeTok{type =} \StringTok{"l"}\NormalTok{, }\AttributeTok{xlab =} \StringTok{"X"}\NormalTok{, }\AttributeTok{ylab =} \StringTok{"dnorm(x,90,10)"}\NormalTok{, }\AttributeTok{lwd =} \DecValTok{2}\NormalTok{)}
\FunctionTok{segments}\NormalTok{(ci[}\DecValTok{1}\NormalTok{], }\DecValTok{0}\NormalTok{, ci[}\DecValTok{1}\NormalTok{], }\FunctionTok{dnorm}\NormalTok{(ci[}\DecValTok{1}\NormalTok{], }\AttributeTok{mean =} \DecValTok{90}\NormalTok{, }\AttributeTok{sd =} \DecValTok{10}\NormalTok{), }\AttributeTok{lwd =} \DecValTok{2}\NormalTok{, }\AttributeTok{lty =} \DecValTok{2}\NormalTok{)}
\FunctionTok{segments}\NormalTok{(ci[}\DecValTok{2}\NormalTok{], }\DecValTok{0}\NormalTok{, ci[}\DecValTok{2}\NormalTok{], }\FunctionTok{dnorm}\NormalTok{(ci[}\DecValTok{2}\NormalTok{], }\AttributeTok{mean =} \DecValTok{90}\NormalTok{, }\AttributeTok{sd =} \DecValTok{10}\NormalTok{), }\AttributeTok{lwd =} \DecValTok{2}\NormalTok{, }\AttributeTok{lty =} \DecValTok{2}\NormalTok{)}
\end{Highlighting}
\end{Shaded}

\includegraphics{block3_files/figure-pdf/unnamed-chunk-34-1.pdf}

Den sortens beräkningar du har gjort med normalfördelningen är mycket
vanliga. Därför finns så kallade z-tabeller publicerade. De anger ofta
sannolikheten, p , att få ett utfall z eller lägre än z, där z är
normalfördelad med medelvärdet 0 och standardavvikelsen 1.

Testa att googla z table och titta på bildresultaten, så hittar du
z-tabeller i olika utformningar.

Du kan använda länken nedan om du vill, men lova då att klura på hur
adressen är utformad, det kan du ha glädje av någon annan gång när du
hämtar sidor automatiskt.

\url{https://www.google.com/search?q=z+table}

\textbf{6.11} Skriv några rader i R som skapar en kolumn med p-värden,
från z= -4 till z = 0 Välj ett antal värdesiffror som du tycker verkar
vara vanligt i z-tabeller. Vi bryr oss inte om smärre avvikelser i
avrundning.

\begin{Shaded}
\begin{Highlighting}[]
\CommentTok{\# Z och P värden}
\NormalTok{z }\OtherTok{\textless{}{-}} \FunctionTok{seq}\NormalTok{(}\SpecialCharTok{{-}}\DecValTok{4}\NormalTok{, }\DecValTok{0}\NormalTok{, }\AttributeTok{by =} \DecValTok{1}\NormalTok{)}
\NormalTok{p\_values }\OtherTok{\textless{}{-}} \FunctionTok{pnorm}\NormalTok{(z)}

\CommentTok{\# Skapa df}
\NormalTok{p\_values\_df }\OtherTok{\textless{}{-}} \FunctionTok{data.frame}\NormalTok{(}\AttributeTok{z =}\NormalTok{ z, }\AttributeTok{p\_values =} \FunctionTok{round}\NormalTok{(p\_values, }\DecValTok{4}\NormalTok{))}

\FunctionTok{print}\NormalTok{(p\_values\_df)}
\end{Highlighting}
\end{Shaded}

\begin{verbatim}
   z p_values
1 -4   0.0000
2 -3   0.0013
3 -2   0.0228
4 -1   0.1587
5  0   0.5000
\end{verbatim}

\textbf{Varför är normalfördelningen användbar?}

Nu har du räknat ett antal exempel med normalfördelningen. Men varför är
normalfördelningen så användbar egentligen? Jag vill att du gör två
experiment som illustrerar varför vi har stor glädje av denna
fördelning. Det första utgår ifrån ett resonemang om biologi och
binomialfördelningen, det andra är mer matematiskt och handlar om
centrala gränsvärdessatsen.

Många fenomen i naturen beror på ett stort antal underliggande faktorer
som är oberoende av varandra. Ofta duger det bra att anta oberoende även
i fall där det finns små kopplingar mellan faktorer. Tänk till exempel
att du ska skapa detaljerad modell av kroppslängd. Troligen kan du hitta
påverkan från ett stort antal gener, vi kan gissa på 200. Dessutom kan
du säkert hitta ett så stort antal yttre faktorer att det verkar rimligt
att klassa dem som oberoende för vårt resonemang. Mammas
nutritionstillstånd under graviditeten. Mamma rökte eller ej. Min poäng
här är bara att listan kan göras lång.

Tänk dig en biologisk variabel x. Du kan simulera ett värde för x genom
att singla slant ett antal gånger och lägga samman resultatet: varje
krona ger 1 poäng, varje klave ger 0 poäng. Använd binomialfördelningen
för att se hur x fördelas i popultaionen om variabeln styrs av två
underliggande faktorer. (7.1)

Jämförelsen haltar något, eftersom vi inte har med koncepten recessiv
och dominant i vår enkla modell, men för att konkretisera kan du tänka
dig en enklare egenskap som ögonfärg. Mycket förenklat kan den anta
några få olika lägen: blå, grön, brun. Och den styrs av få underliggande
faktorer: ett par gener.

\textbf{7.1} Rita upp binomialfördelningen för att singla slant två ggr
med följande kod.

Rita in en normalfördelning i samma graf.

\begin{Shaded}
\begin{Highlighting}[]
\NormalTok{x }\OtherTok{\textless{}{-}} \DecValTok{0}\SpecialCharTok{:}\DecValTok{2}
\FunctionTok{plot}\NormalTok{(x, }\FunctionTok{dbinom}\NormalTok{(x, }\DecValTok{2}\NormalTok{, }\FloatTok{0.5}\NormalTok{), }\AttributeTok{type =} \StringTok{"h"}\NormalTok{, }\AttributeTok{col =} \StringTok{"blue"}\NormalTok{, }\AttributeTok{lwd=}\DecValTok{4}\NormalTok{, }\AttributeTok{ylim=} \FunctionTok{c}\NormalTok{(}\DecValTok{0}\NormalTok{,}\FloatTok{0.6}\NormalTok{))}

\NormalTok{mean }\OtherTok{\textless{}{-}} \DecValTok{2} \SpecialCharTok{*} \FloatTok{0.5}
\NormalTok{sd }\OtherTok{\textless{}{-}} \FunctionTok{sqrt}\NormalTok{(}\DecValTok{2} \SpecialCharTok{*} \FloatTok{0.5} \SpecialCharTok{*} \FloatTok{0.5}\NormalTok{)}

\NormalTok{x\_norm }\OtherTok{\textless{}{-}} \FunctionTok{seq}\NormalTok{(}\SpecialCharTok{{-}}\DecValTok{1}\NormalTok{, }\DecValTok{3}\NormalTok{, }\AttributeTok{length.out =} \DecValTok{100}\NormalTok{)}
\NormalTok{norm\_y }\OtherTok{\textless{}{-}} \FunctionTok{dnorm}\NormalTok{(x\_norm, mean, sd)}

\FunctionTok{lines}\NormalTok{(x\_norm, norm\_y, }\AttributeTok{col =} \StringTok{"red"}\NormalTok{, }\AttributeTok{lwd =} \DecValTok{2}\NormalTok{)}
\end{Highlighting}
\end{Shaded}

\includegraphics{block3_files/figure-pdf/unnamed-chunk-36-1.pdf}

\textbf{7.2} Rita nu upp en variabel som styrs av 8 underliggande
oberoende faktorer Rita in en normalfördelning i samma graf.

\begin{Shaded}
\begin{Highlighting}[]
\CommentTok{\# Variabler}
\NormalTok{n }\OtherTok{\textless{}{-}} \DecValTok{10000}
\NormalTok{factors }\OtherTok{\textless{}{-}} \FunctionTok{replicate}\NormalTok{(}\DecValTok{8}\NormalTok{, }\FunctionTok{rnorm}\NormalTok{(n))}
\NormalTok{variable }\OtherTok{\textless{}{-}} \FunctionTok{rowSums}\NormalTok{(factors)}

\CommentTok{\# Plot the histogram of the variable}
\FunctionTok{hist}\NormalTok{(variable, }\AttributeTok{breaks =} \DecValTok{50}\NormalTok{, }\AttributeTok{probability =} \ConstantTok{TRUE}\NormalTok{)}

\CommentTok{\# Overlay a normal distribution curve}
\NormalTok{x }\OtherTok{\textless{}{-}} \FunctionTok{seq}\NormalTok{(}\FunctionTok{min}\NormalTok{(variable), }\FunctionTok{max}\NormalTok{(variable), }\AttributeTok{length =} \DecValTok{100}\NormalTok{)}
\NormalTok{y }\OtherTok{\textless{}{-}} \FunctionTok{dnorm}\NormalTok{(x, }\AttributeTok{mean =} \FunctionTok{mean}\NormalTok{(variable), }\AttributeTok{sd =} \FunctionTok{sd}\NormalTok{(variable))}
\FunctionTok{lines}\NormalTok{(x, y, }\AttributeTok{col =} \StringTok{"red"}\NormalTok{, }\AttributeTok{lwd =} \DecValTok{2}\NormalTok{)}
\end{Highlighting}
\end{Shaded}

\includegraphics{block3_files/figure-pdf/unnamed-chunk-37-1.pdf}

\textbf{7.3} Rita nu upp en variabel som styrs av 30 underliggande
oberoende faktorer Rita in en normalfördelning i samma graf.

\begin{Shaded}
\begin{Highlighting}[]
\CommentTok{\# Variabler}
\NormalTok{n }\OtherTok{\textless{}{-}} \DecValTok{10000}
\NormalTok{factors }\OtherTok{\textless{}{-}} \FunctionTok{replicate}\NormalTok{(}\DecValTok{30}\NormalTok{, }\FunctionTok{rnorm}\NormalTok{(n))}
\NormalTok{variable }\OtherTok{\textless{}{-}} \FunctionTok{rowSums}\NormalTok{(factors)}

\CommentTok{\# Plot}
\FunctionTok{hist}\NormalTok{(variable, }\AttributeTok{breaks =} \DecValTok{50}\NormalTok{, }\AttributeTok{probability =} \ConstantTok{TRUE}\NormalTok{)}

\CommentTok{\# Overlay}
\NormalTok{x }\OtherTok{\textless{}{-}} \FunctionTok{seq}\NormalTok{(}\FunctionTok{min}\NormalTok{(variable), }\FunctionTok{max}\NormalTok{(variable), }\AttributeTok{length =} \DecValTok{100}\NormalTok{)}
\NormalTok{y }\OtherTok{\textless{}{-}} \FunctionTok{dnorm}\NormalTok{(x, }\AttributeTok{mean =} \FunctionTok{mean}\NormalTok{(variable), }\AttributeTok{sd =} \FunctionTok{sd}\NormalTok{(variable))}
\FunctionTok{lines}\NormalTok{(x, y, }\AttributeTok{col =} \StringTok{"red"}\NormalTok{, }\AttributeTok{lwd =} \DecValTok{2}\NormalTok{)}
\end{Highlighting}
\end{Shaded}

\includegraphics{block3_files/figure-pdf/unnamed-chunk-38-1.pdf}

\textbf{7.4} Kommentera resultatet av undersökningen 7.1 till 7.3 Även
summerade variabler beroende på 8 och 30 faktorer uppvisar normal
distribution, oavsett fördelning på underliggande faktorer. Det innebär
även att vi kan anta att variabler som påverkas av underliggande
faktorer uppvisar normal distribution, oavsett dess underliggande
distribution.

\textbf{Frivillig uppgift}: Det finns ett elegant sätt att välja
standardavvikelse för normalfördelningen som du ritar in över
binomialfördelningen i graferna. Hur kan man beräkna ett lämpligt värde?
Tips: Läs Lind formel {[}6-5{]}

\textbf{Centrala gränsvärdessatsen (Central Limit theorem)}

Att annat skäl till att normalfördelningen är så användbar inom
statistiken följer av centrala gränsvärdessatsen. Tänk att du drar många
lika stora stickprov ur en population och varje gång beräknar
stickprovets medelvärde. Vi ska nu undersöka hur stickprovets medelvärde
fördelas.

\textbf{8.1} Skapa ett antal fördelningar NORM, UNIF, SKEV med koden
nedan Använd gruppens nummer som seed, tex grupp A = seed(1), grpp B,
seed(2)

\begin{Shaded}
\begin{Highlighting}[]
\FunctionTok{set.seed}\NormalTok{(}\DecValTok{10}\NormalTok{)}
\NormalTok{NORM }\OtherTok{\textless{}{-}} \FunctionTok{rnorm}\NormalTok{(}\DecValTok{10000}\NormalTok{)}
\NormalTok{UNIF }\OtherTok{\textless{}{-}} \FunctionTok{runif}\NormalTok{(}\DecValTok{10000}\NormalTok{)}
\NormalTok{SKEV }\OtherTok{\textless{}{-}} \FunctionTok{rep}\NormalTok{(}\DecValTok{1}\SpecialCharTok{:}\DecValTok{100}\NormalTok{, }\DecValTok{1}\SpecialCharTok{:}\DecValTok{100}\NormalTok{)}
\end{Highlighting}
\end{Shaded}

\textbf{8.2} Rita histogram över fördelningarna och beräkna medelvärde
och standardavvikelse (även om man kan ifrågasätta iden att räkna medel
och standardavvikelse för SKEV)

\begin{Shaded}
\begin{Highlighting}[]
\FunctionTok{hist}\NormalTok{(NORM)}
\FunctionTok{abline}\NormalTok{(}\AttributeTok{v =} \FunctionTok{mean}\NormalTok{(NORM), }\AttributeTok{col =} \StringTok{"red"}\NormalTok{, }\AttributeTok{lwd =} \DecValTok{2}\NormalTok{, }\AttributeTok{lty =} \DecValTok{2}\NormalTok{)}
\FunctionTok{abline}\NormalTok{(}\AttributeTok{v =} \FunctionTok{mean}\NormalTok{(NORM) }\SpecialCharTok{+} \FunctionTok{sd}\NormalTok{(NORM), }\AttributeTok{col =} \StringTok{"blue"}\NormalTok{, }\AttributeTok{lwd =} \DecValTok{2}\NormalTok{, }\AttributeTok{lty =} \DecValTok{2}\NormalTok{)}
\FunctionTok{abline}\NormalTok{(}\AttributeTok{v =} \FunctionTok{mean}\NormalTok{(NORM) }\SpecialCharTok{{-}} \FunctionTok{sd}\NormalTok{(NORM), }\AttributeTok{col =} \StringTok{"blue"}\NormalTok{, }\AttributeTok{lwd =} \DecValTok{2}\NormalTok{, }\AttributeTok{lty =} \DecValTok{2}\NormalTok{)}
\end{Highlighting}
\end{Shaded}

\includegraphics{block3_files/figure-pdf/unnamed-chunk-40-1.pdf}

\begin{Shaded}
\begin{Highlighting}[]
\FunctionTok{hist}\NormalTok{(UNIF)}
\FunctionTok{abline}\NormalTok{(}\AttributeTok{v =} \FunctionTok{mean}\NormalTok{(UNIF), }\AttributeTok{col =} \StringTok{"red"}\NormalTok{, }\AttributeTok{lwd =} \DecValTok{2}\NormalTok{, }\AttributeTok{lty =} \DecValTok{2}\NormalTok{)}
\FunctionTok{abline}\NormalTok{(}\AttributeTok{v =} \FunctionTok{mean}\NormalTok{(UNIF) }\SpecialCharTok{+} \FunctionTok{sd}\NormalTok{(UNIF), }\AttributeTok{col =} \StringTok{"blue"}\NormalTok{, }\AttributeTok{lwd =} \DecValTok{2}\NormalTok{, }\AttributeTok{lty =} \DecValTok{2}\NormalTok{)}
\FunctionTok{abline}\NormalTok{(}\AttributeTok{v =} \FunctionTok{mean}\NormalTok{(UNIF) }\SpecialCharTok{{-}} \FunctionTok{sd}\NormalTok{(UNIF), }\AttributeTok{col =} \StringTok{"blue"}\NormalTok{, }\AttributeTok{lwd =} \DecValTok{2}\NormalTok{, }\AttributeTok{lty =} \DecValTok{2}\NormalTok{) }
\end{Highlighting}
\end{Shaded}

\includegraphics{block3_files/figure-pdf/unnamed-chunk-40-2.pdf}

\begin{Shaded}
\begin{Highlighting}[]
\FunctionTok{hist}\NormalTok{(SKEV)}
\FunctionTok{abline}\NormalTok{(}\AttributeTok{v =} \FunctionTok{mean}\NormalTok{(SKEV), }\AttributeTok{col =} \StringTok{"red"}\NormalTok{, }\AttributeTok{lwd =} \DecValTok{2}\NormalTok{, }\AttributeTok{lty =} \DecValTok{2}\NormalTok{)}
\FunctionTok{abline}\NormalTok{(}\AttributeTok{v =} \FunctionTok{mean}\NormalTok{(SKEV) }\SpecialCharTok{+} \FunctionTok{sd}\NormalTok{(SKEV), }\AttributeTok{col =} \StringTok{"blue"}\NormalTok{, }\AttributeTok{lwd =} \DecValTok{2}\NormalTok{, }\AttributeTok{lty =} \DecValTok{2}\NormalTok{)}
\FunctionTok{abline}\NormalTok{(}\AttributeTok{v =} \FunctionTok{mean}\NormalTok{(SKEV) }\SpecialCharTok{{-}} \FunctionTok{sd}\NormalTok{(SKEV), }\AttributeTok{col =} \StringTok{"blue"}\NormalTok{, }\AttributeTok{lwd =} \DecValTok{2}\NormalTok{, }\AttributeTok{lty =} \DecValTok{2}\NormalTok{)}
\end{Highlighting}
\end{Shaded}

\includegraphics{block3_files/figure-pdf/unnamed-chunk-40-3.pdf}

\textbf{8.3} Börja arbeta med NORM. Tag ur NORM 1000 stickprov med
återläggning av storleken n=3, tag sedan 1000 stickprov av storleken
n=6, tag slutligen 1000 stickprov av storleken n=300.

\begin{Shaded}
\begin{Highlighting}[]
\CommentTok{\# Funktion för plotting}
\NormalTok{plot\_sample\_means }\OtherTok{\textless{}{-}} \ControlFlowTok{function}\NormalTok{(data, sample\_sizes, num\_samples) \{}
  \CommentTok{\# Nestad funktion för att beräkna medelvärde och sd}
\NormalTok{  sample\_means }\OtherTok{\textless{}{-}} \ControlFlowTok{function}\NormalTok{(data, sample\_size, num\_samples) \{}
\NormalTok{    means }\OtherTok{\textless{}{-}} \FunctionTok{numeric}\NormalTok{(num\_samples)}
    \ControlFlowTok{for}\NormalTok{ (i }\ControlFlowTok{in} \DecValTok{1}\SpecialCharTok{:}\NormalTok{num\_samples) \{}
\NormalTok{      sample }\OtherTok{\textless{}{-}} \FunctionTok{sample}\NormalTok{(data, sample\_size, }\AttributeTok{replace =} \ConstantTok{TRUE}\NormalTok{)}
\NormalTok{      means[i] }\OtherTok{\textless{}{-}} \FunctionTok{mean}\NormalTok{(sample)}
\NormalTok{    \}}
    \FunctionTok{return}\NormalTok{(means)}
\NormalTok{  \}}
  
  \FunctionTok{par}\NormalTok{(}\AttributeTok{mfrow =} \FunctionTok{c}\NormalTok{(}\FunctionTok{length}\NormalTok{(sample\_sizes), }\DecValTok{1}\NormalTok{))  }\CommentTok{\# Set up the plotting area}
  
  \ControlFlowTok{for}\NormalTok{ (n }\ControlFlowTok{in}\NormalTok{ sample\_sizes) \{}
\NormalTok{    means }\OtherTok{\textless{}{-}} \FunctionTok{sample\_means}\NormalTok{(data, n, num\_samples)}
    \FunctionTok{hist}\NormalTok{(means, }\AttributeTok{main =} \FunctionTok{paste}\NormalTok{(}\StringTok{"(n ="}\NormalTok{, n, }\StringTok{")"}\NormalTok{))}
    \FunctionTok{abline}\NormalTok{(}\AttributeTok{v =} \FunctionTok{mean}\NormalTok{(means), }\AttributeTok{col =} \StringTok{"red"}\NormalTok{, }\AttributeTok{lwd =} \DecValTok{2}\NormalTok{, }\AttributeTok{lty =} \DecValTok{2}\NormalTok{)}
    \FunctionTok{abline}\NormalTok{(}\AttributeTok{v =} \FunctionTok{mean}\NormalTok{(means) }\SpecialCharTok{+} \FunctionTok{sd}\NormalTok{(means), }\AttributeTok{col =} \StringTok{"blue"}\NormalTok{, }\AttributeTok{lwd =} \DecValTok{2}\NormalTok{, }\AttributeTok{lty =} \DecValTok{2}\NormalTok{)}
    \FunctionTok{abline}\NormalTok{(}\AttributeTok{v =} \FunctionTok{mean}\NormalTok{(means) }\SpecialCharTok{{-}} \FunctionTok{sd}\NormalTok{(means), }\AttributeTok{col =} \StringTok{"blue"}\NormalTok{, }\AttributeTok{lwd =} \DecValTok{2}\NormalTok{, }\AttributeTok{lty =} \DecValTok{2}\NormalTok{)}
\NormalTok{  \}}
  
  \FunctionTok{par}\NormalTok{(}\AttributeTok{mfrow =} \FunctionTok{c}\NormalTok{(}\DecValTok{1}\NormalTok{, }\DecValTok{1}\NormalTok{))  }\CommentTok{\# Reset plotting area}
\NormalTok{\}}

\NormalTok{sample\_sizes }\OtherTok{\textless{}{-}} \FunctionTok{c}\NormalTok{(}\DecValTok{3}\NormalTok{, }\DecValTok{6}\NormalTok{, }\DecValTok{300}\NormalTok{)}
\NormalTok{num\_samples }\OtherTok{\textless{}{-}} \DecValTok{1000}
\end{Highlighting}
\end{Shaded}

\begin{Shaded}
\begin{Highlighting}[]
\FunctionTok{plot\_sample\_means}\NormalTok{(NORM, sample\_sizes, num\_samples)}
\end{Highlighting}
\end{Shaded}

\includegraphics{block3_files/figure-pdf/unnamed-chunk-42-1.pdf}

Beräkna medelvärde för varje stickprov och rita histogram över
medelvärdena (du har gjort en empirisk samplingsfördelning). Beräkna
medelvärdet och standardavvikelsen för dina samplingsfördelningar n=3,
n=6, n=300 .

\textbf{8.4} Upprepa uppgift 8.3 med UNIF och till sist med SKEV

\begin{Shaded}
\begin{Highlighting}[]
\FunctionTok{plot\_sample\_means}\NormalTok{(UNIF, sample\_sizes, num\_samples)}
\end{Highlighting}
\end{Shaded}

\includegraphics{block3_files/figure-pdf/unnamed-chunk-43-1.pdf}

\begin{Shaded}
\begin{Highlighting}[]
\FunctionTok{plot\_sample\_means}\NormalTok{(SKEV, sample\_sizes, num\_samples)}
\end{Highlighting}
\end{Shaded}

\includegraphics{block3_files/figure-pdf/unnamed-chunk-44-1.pdf}

\textbf{8.5} Formulera centrala gränsvärdessatsen (CGS), citera gärna ur
en bok (tex Lind) eller från internet. Kommentera resultatet i 8.2 till
8.4 med hjälp av CGS.

För NORM (normalfördelning): Provmitten är normalfördelade, vilket är
väntat eftersom den ursprungliga fördelningen är normal.

För UNIF (uniform fördelning): Trots att den ursprungliga fördelningen
är uniform närmar sig fördelningen av provmedelvärdena en
normalfördelning när provstorleken ökar. Detta visar på den centrala
gränsvärdessatsen, som säger att provmedelvärdena kommer att vara
normalfördelade oavsett den ursprungliga fördelningen.

För SKEV (skev fördelning): Även om den ursprungliga fördelningen är
skev blir fördelningen av provmedelvärdena ungefär normal när
provstorleken ökar. Detta illustrerar återigen den centrala
gränsvärdessatsen, som visar att provmedelvärdena tenderar att vara
normalfördelade oavsett skevheten i den ursprungliga fördelningen.




\end{document}
